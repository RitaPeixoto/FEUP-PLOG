\section{Introdução}

No âmbito da unidade curricular de Programação em Lógica, foi proposta a construção de um programa na linguagem Prolog para a resolução de um problema de decisão, com recurso à programação em lógica com restrições, utilizando a biblioteca clpfd. 

O grupo escolheu o puzzle Gold Star, que consiste num conjunto de equações válidas exibidas em forma de estrela. 
Este trabalho divide-se em duas grandes partes que são a pesquisa da solução para problema e a geração aleatória de problemas.

O presente artigo começa por descrever o problema, passando para a abordagem na resolução do mesmo, onde é possível encontrar as variáveis de decisão, as restrições e a geração de problemas. Após isto aborda-se a visualização das soluções, é feita a análise dimensional do problema e são analisadas diferentes estratégias de pesquisa. Terminando com as conclusões a partir dos dados obtidos.