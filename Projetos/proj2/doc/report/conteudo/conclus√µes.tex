\section{Conclusões e trabalho futuro}
Este trabalho contribuiu para aplicação dos conhecimentos teóricos de programação em lógica com restrições abrangendo diversos tópicos. 

Note-se que a solução apresentada poderia ser melhorada e optimizada, de modo a torná-la mais eficiente. Com o aumento da dimensão, o tempo de execução aumenta consideravelmente. 

No desenvolvimento deste trabalho foram encontradas dificuldades em remover simetrias e reflexões, tendo sido feitos alguns testes sem quaisquer efeitos, uma vez que o problema tem uma forma complexa. Pensa-se que a restrição aplicada surte bastante efeito, sendo possível haver melhores restrições para diminuir ainda mais o espaço de resposta, mas que o grupo não encontrou, deixando para trabalho futuro.


Conclui-se que o trabalho foi concluído com sucesso, foi possível encontrar tanto soluções para problema base, bem como para problemas de diferentes dimensões e foi possível gerar aleatoriamente problemas a serem resolvidos.
